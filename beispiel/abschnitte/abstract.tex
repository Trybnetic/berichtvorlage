Unsere Aufmerksamkeit ist schon lange ein zentrales Forschungsthema der Psychologie. Heute geht man davon aus, dass es sich bei der Aufmerksamkeit um drei funktional verschiedene Systeme handelt, der bewussten Aufmerksamkeitslenkung, der Aufmerksamkeitsaktivierung und der räumlichen Aufmerksamkeitsausrichtung.
Neuere Studien zeigten, dass das System der Aufmerksamkeitsaktivierung und das System der Aufmerksamkeitslenkung unter bestimmten Umständen interagieren \cite{weinbach2012relationship}. 
In unserer Studie versuchten wir, diese Interaktion näher zu untersuchen. 
Unser Ziel war es herauszufinden, ob die Aufmerksamkeitsaktivierung zu einer reduzierten Abnahme des Kongruenzeffekts bei größer werdender Distanz führt. 
Um unsere Hypothese zu überprüfen, erweiterten wir ein Experiment von \citeA{weinbach2012relationship} um den Faktor der Distanz. 
Die erwartete dreifach Interaktion zwischen Distanz, Kongruenz und Ton konnten wir nicht nachweisen. 
Auch die Replikation des Experiments von \citeA{weinbach2012relationship} zeigte lediglich signifikante Haupteffekte für Kongruenz und Ton. 
Eine signifikante Interaktion der beiden Faktoren konnten wir nicht replizieren. 
