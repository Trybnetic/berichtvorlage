Aufmerksamkeit ist eine wichtige kognitive Fähigkeit. Sie ermöglicht es uns, bestimmte Reize verstärkt zu verarbeiten und ermöglicht so zielgerichtetes Handeln. Zahlreiche Untersuchungen haben gezeigt, dass sich unsere Aufmerksamkeit in drei unterschiedliche Systeme unterscheiden lässt. Es wurde vermutet, dass diese Systeme unabhängig voneinander funktionieren. Neuere Studien legten jedoch nahe, dass das System der bewussten Aufmerksamkeitslenkung und das System der Aufmerksamkeitsaktivierung unter bestimmten Umständen miteinander interagieren. Wie diese Interaktion funktioniert, konnte bisher nicht hinreichend erklärt werden. Als eine mögliche Erklärung wurde vorgeschlagen, dass sich der räumliche Aufmerksamkeitsfokus durch diese Interaktion erweitert. Diesen Effekt wollen wir in dieser Studie genauer Untersuchen.\\
Die bewusste Aufmerksamkeitslenkung ist notwendig, um relevante von irrelevanten Reizen zu unterscheiden. Sie wird meist durch Konfliktreizaufgaben, wie Flankierreizexperimenten \cite{eriksen1974effects} oder Stroop-Aufgaben \cite{stroop1935studies} untersucht. Gemessen wird sie durch eine verkürzte Reaktionszeit bei kongruenten Bedingungen, die zusätzlichen Reize führen zu keinem Konflikt mit dem Zielreiz, im Vergleich zu inkongruenten Bedingungen, die zusätzlichen Reize führen zu einem Konflikt mit dem Zielreiz.\\
Ein weiteres System der Aufmerksamkeit ist die Aufmerksamkeitsaktivierung. Sie lässt sich in die tonische und die phasische Aufmerksamkeitsaktivierung unterschieden. Die tonische Aufmerksamkeitsaktivierung wird bewusst und ohne externen Reiz nach dem Top-Down Prinzip gesteuert. Die phasische Aufmerksamkeitsaktivierung wird durch einen Warnreiz ausgelöst und ist von kurzer Dauer, erreicht aber ein hohes Level von Aufmerksamkeit. Gemessen wird die phasische Aufmerksamkeitsaktivierung durch eine verkürzte Reaktionszeit in den Versuchsdurchgängen mit einem Warnreiz im Vergleich zu den Durchgängen, in denen kein Warnreiz präsentiert wird.\\
Das Dritte, aber für unsere Untersuchung nicht relevante Aufmerksamkeitssystem, ist das System der Aufmerksamkeitsausrichtung. Untersucht wird sie mit Hinweisreizaufgaben. Die Probanden erhalten dabei vor der Präsentation des Zielreizes einen Hinweisreiz.\\ 
Diese Systeme wurden schon in vielen Studien unabhängig voneinander untersucht, doch neuere Untersuchungen legen nahe, dass das System der Aufmerksamkeitsaktivierung und das System der bewussten Aufmerksamkeitslenkung unter bestimmten Umständen miteinander interagieren \cite{callejas2004three,weinbach2012relationship}. \textcite{weinbach2012relationship} nutzten in ihrer Studie eine Weiterentwicklung des ''attentional network test'', kurz ANT \cite{fan2002testing}. Der ANT wurde entwickelt, um alle drei Aufmerksamkeitssysteme gleichzeitig zu untersuchen. Im ANT kombinierten \textcite{fan2002testing} eine Eriksen-Aufgabe, zur Messung der bewussten Aufmerksamkeitslenkung, mit einem Warnreiz, zur Messung der Aufmerksamkeitsaktivierung, und räumlichen Hinweisreizen, zur Messung der räumlichen Aufmerksamkeitslenkung.\\
Im vierten Experiment ihrer Untersuchung reduzierten \parencite{weinbach2012relationship} den ANT um den räumlichen Hinweisreiz und erweiterten ihn um eine Stroop-Aufgabe. Den Probanden wurde ein Zielreiz, welcher entweder Rot oder Grün war, präsentiert, der sowohl links als auch rechts von Pfeilen flankiert wurde. Die Aufgabe der Probanden war es, auf einen grünen Zielreiz mit der rechten Hand und auf einen roten Zielreiz mit der linken Hand zu reagieren. Die Probanden sollten dabei die Richtung der Pfeile, die entweder alle nach links, nach rechts oder in beide Richtungen zeigen konnten, ignorieren. Zur Messung der Aufmerksamkeitsaktivierung wurde in der Hälfte der Trials den Versuchpersonen ein auditiver Reiz vor der Darbietung des Zielreizes präsentiert. \textcite{weinbach2012relationship} fanden dabei einen signifikanten Interaktionseffekt zwischen Aufmerksamkeitsaktivierung und Aufmerksamkeitslenkung. Als mögliche Erklärung für diese Interaktion schlugen sie eine Erweiterung des Aufmerksamkeitsfokus durch den Hinweisreiz vor.\\
Ziel unserer Studie ist es, diese Interaktion weiter zu erforschen und die Theorie von \textcite{weinbach2012relationship} zu überprüfen. Wir replizieren dazu den vierten Versuch von \textcite{weinbach2012relationship} und manipulieren zusätzlich die Distanz zwischen Zielreiz und Flankierreizen und  zwischen den Flankierreizen. Unsere Hypothese ist es, dass der präsentierte Ton die Abnahme des Kongruenzeffekts mit der Distanz reduziert. Des Weiteren erwarten wir, die Ergebnisse von \textcite{weinbach2012relationship} zu replizieren, also Haupteffekte für Kongruenz und Aufmerksamkeitsaktivierung sowie einen Interaktionseffekt der beiden Faktoren zu finden. Auch einen Haupteffekt für die Distanz sowie eine Interaktion zwischen Distanz und Kongruenz, wie ihn frühere Studien schon zeigten, hoffen wir zu finden \cite{eriksen1974effects}.