Reaktionszeiten, die kleiner waren als 200 ms oder größer waren als 1.000 ms, wurden von der Analyse ausgelassen (dies betraf weniger als 3\% der Trials). Der Anteil der korrekten Trials betrug 97\%. Die Reaktionszeiten für korrekte Versuchsdurchgänge in den einzelnen Versuchsbedingungen wurde analysiert mit den unabhängigen Faktoren Ton (mit Ton, ohne Ton), Kongruenz (kongruente Bedingung, inkongruente Bedingung) und Distanz (0.05°, 0.1°, 0.2°).\\
Die Analyse zeigte Haupteffekte für Kongruenz und Ton, $F(1,22)=25.20$, $p<.001$ und $F(1,22)=29.17$, $p<.001$. Die Interaktion zwischen Ton und Kongruenz war nicht signifikant, $F(1,22)=1.57$, $p=.22$. Die Interaktionen zwischen der Distanz und der Kongruenz und zwischen der Distanz und Ton waren beide nicht signifikant, $F(2,44)=0.99$, $p=.38$ und $F(2,44)=0.43$, $p=.65$. Die dreifache Interaktion zwischen Kongruenz, Ton und Distanz war ebenfalls nicht signifikant, $F(2,44)=0.63$, $p=.54$.
